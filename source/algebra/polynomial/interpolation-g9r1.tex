% $date: 2016-03-29
% $timetable:
%   g9r1:
%     2016-03-29:
%       2:

\section*{Интерполяция}

% $authors:
% - Антон Сергеевич Гусев

\begin{problems}

\item
Значениями в~скольких точках многочлен степени~$n$ задается однозначно?

\end{problems}

\definition
Построение такого многочлена $P(x)$ степени не~выше $n$, что
$P(x_{i}) = y_{i}$, где $0\leq i \leq n$, называется \emph{интерполяцией.}

\begin{problems}

\item
Выведите \textbf{интерполяционную формулу Лагранжа}
\[
    P(x)
=
    \sum_{i=0}^{n}
        y_i
        \frac{x - x_{0}}{x_{i} - x_{0}}
        \cdot \ldots \cdot
        \frac{x - x_{i-1}}{x_{i} - x_{i-1}}
        \cdot
        \frac{x - x_{i+1}}{x_i - x_{i+1}}
        \cdot \ldots \cdot
        \frac{x-x_n}{x_i-x_n}
\; . \]

\item
Докажите, что если $a$, $b$, $c$, $d$ различны, то
\begin{align*} &
    \frac{(x - a)(x - b)(x - c)}{(d - a)(d - b)(d - c)} +
    \frac{(x - a)(x - b)(x - d)}{(c - a)(c - b)(c - d)}
+ \\ & \quad +
    \frac{(x - a)(x - c)(x - d)}{(b - a)(b - c)(b - d)} +
    \frac{(x - b)(x - c)(x - d)}{(a - b)(a - c)(a - d)}
=
    1
\, . \end{align*}

\item
Многочлен $Q(x)$ степени~$n$ принимает целые значения при
$x = 0, 1, 4, \ldots, n^2$.
Докажите что, многочлен $Q(x)$ принимает целые значения при всех квадратах
натуральных чисел.

\item
Вася задумал многочлен десятой степени.
Петя может назвать десять вещественных чисел и~Вася сообщит ему значение
многочлена  при одном из~названных значений переменной.
При этом Вася не~сообщает, какое именно число из~названных Петей он подставил.
\\
\subproblem
Может~ли Петя определить Васин многочлен за~несколько вопросов?
\\
\subproblem
Если да, то~какое наименьшее число вопросов ему для этого потребуется?

\item
Пусть $P(x)$~--- многочлен степени не~выше $n$, для которого $P(i) = 2^i$ при
$i = 0, 1, \ldots, n$.
Найдите $P(n + 1)$.

\item
$P(x)$~--- многочлен $n$-й степени.
$P(0) = 1$, $P(1) = 1 / 2$, $P(2) = 1 / 3$,~\ldots, $P(n) = 1 / (n + 1)$.
Найдите
\\
\subproblem $P(n + 1)$;
\qquad
\subproblem $P(3n)$.

\item
Существует~ли многочлен с~целыми коэффициентами $P(x)$ степени~$n$ такой, что
$P(0)$,~\ldots, $P(n)$~--- различные степени двойки?

\end{problems}

