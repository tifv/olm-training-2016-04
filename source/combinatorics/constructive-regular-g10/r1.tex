% $date: 2016-04-06
% $timetable:
%   g10r1:
%     2016-04-06:
%       2:

\section*{Конструктивы. Регулярные примеры}

% $authors:
% - Владимир Алексеевич Брагин
% - Иван Викторович Митрофанов

\subsection*{Графы}

\begin{problems}

\item
В~Марсианской Футбольной Премьер Лиге $2016$ команд.
Руководство лиги хочет провести турнир из~$2015$ туров, так чтобы в~каждом туре
каждая команда играла ровно один матч.
Кроме того, после всех туров любые две команды должны сыграть между собой ровно
один раз.
Докажите, что это возможно.
% Сколько есть способов так сделать?

\item
На~чемпионат по~шведкам приехало $n$~людей.
В~каждом матче произвольные две пары людей играют друг против друга.
При каких $n$ можно организовать расписание матчей так, чтобы после турнира
любые двое ровно однажды побывали соперниками
(то~есть просто в~парах друг против друга, необязательно за~одной доской)?

\itemx{*}
Назовем граф \emph{замечательным,} если в~нем степень каждой вершины равна $k$,
у~любых двух смежных вершин нет общих соседей, а~у~любых двух несмежных вершин
ровно один общий сосед.
Докажите, что для $k = 7$ замечательный граф существует.

\end{problems}

\subsection*{Кубы}

\begin{problems}

\item
В~стране $16$~городов.
Можно~ли установить между ними дорожное сообщение так, чтобы их каждого города
выходило не~более $5$~дорог и~между любыми двумя городами был путь из~не~более
чем двух дорог?

\item
$16$~команд участвуют в~футбольном первенстве.
Оно проходит в~несколько этапов.
На~каждом этапе какие-то 6 команд играют между собой однокруговой турнир.
Могло~ли оказаться, что после нескольких таких этапов все команды сыграют друг
с~другом по~$2$ раза?

\item
Можно~ли выбрать в~$7$-элементном множестве $7$ четырехэлементных подмножеств
так, чтобы любые два из~них пересекались по~двум элементам?

\item
Можно~ли в~$15$-элементном множестве выбрать $15$ семиэлементных подмножеств
так, чтобы любые два из~них пересекались по~трем элементам, а~любые три
пересекались не~более чем по~одному?

\end{problems}

\subsection*{Догадайтесь сами}

\begin{problems}

\item
Какое наибольшее количество клеток можно отметить в~квадрате
\\
\subproblem $(p^{2} + p + 1) \times (p^{2} + p + 1)$
\\
\subproblem $(p^{4} + p^2 + 1) \times (p^{4} + p^2 +1)$,
\\
чтобы никакие четыре отмеченные клетки не~образовывали прямоугольник?

\end{problems}

