% $date: 2016-03-30
% $timetable:
%   g10r2:
%     2016-03-30:
%       1:

\section*{Подсчёт двумя способами}

% $authors: # in surname order
% - Владимир Алексеевич Брагин
% - Иван Викторович Митрофанов

\begingroup
    \ifdefined\mathup
        \def\piconst{\mathup{\pi}}%
    \else
        \def\piconst{\uppi}%
    \fi

\begin{problems}

\item
У~многочлена $(x^2 + 3 x + 2)^{2016}$ раскрыли скобки и~нашли разность сумм
коэффициентов при четных и~нечетных степенях $x$.
Какое число получилось?

\item
Внутри квадрата отмечено $100$ точек.
Квадрат разбит на~треугольники таким образом, что вершинами треугольников
являются только отмеченные $100$ точек и~вершины квадрата, причем для любого
треугольника из~разбиения каждая отмеченная точка либо лежит вне этого
треугольника, либо является его вершиной (разбиения такого типа называются
триангуляциями).
Найдите число треугольников в~разбиении.

\item
Можно~ли расставить по~кругу семь целых неотрицательных чисел так, чтобы сумма
каких-то трех расположенных подряд чисел была равна $1$, каких-то трех подряд
расположенных~--- $2$,~\ldots, каких-то трех подряд расположенных~--- $7$?

\item
В~каждой клетке бесконечной полоски написано какое-то вещественное число, числа
могут быть разные.
Известно, что сумма чисел в~любых $32$ клетках подряд не~меньше, чем $1024$.
Докажите, что найдутся $63$ клетки подряд с~суммой не~меньше, чем $2016$.

\item
Может~ли во~время шахматной партии на~каждой из~$30$ диагоналей оказаться
нечетное число фигур?

\item
По~кругу расставлены красные и~синие числа.
Каждое красное число равно сумме соседних чисел, а~каждое синее~--- полусумме
соседних чисел.
Докажите, что сумма красных чисел равна нулю.

\item
На~окружности отметили $n$~точек, разбивающие ее на~$n$~дуг.
Окружность повернули вокруг центра на~угол $2 \piconst k / n$
(при некотором натуральном~$k$), в~результате чего отмеченные точки перешли
в~$n$ новых точек, разбивающих окружность на~$n$ новых дуг.
Докажите, что найдется новая дуга, которая целиком лежит в~одной из~старых дуг.
(Считается, что концы дуги ей принадлежат.)

\item
В~стране $2000$ городов.
Каждый город связан беспосадочными двусторонними авиалиниями с~некоторыми
другими городами, причем для каждого города число исходящих из~него авиалиний
есть степень двойки (т.~е. $1, 2, 4, 8, \ldots$). % XXX
Для каждого города~$A$ статистик подсчитал количество маршрутов, имеющих
не~более одной пересадки, связывающих $A$ с~другими городами, а~затем
просуммировал полученные результаты по~всем $2000$ городам.
У~него получилось $100000$.
Докажите, что статистик ошибся.

\item
Есть два ожерелья, в~каждом ожерелье по~$30$ черных и~$15$ белых бусинок.
Оксана хочет приложить второе ожерелье к~первому (разрешается поворачивать
и~переворачивать) так, чтобы как можно больше бусинок совпало по~цвету.
Какое число совпадающих бусинок Оксана может гарантированно получить?

\item
$300$ бюрократов разбиты на~три комиссии по~$100$ человек.
Докажите, что найдутся два таких бюрократа из~разных комиссий, что в~третьей
комиссии есть либо $17$~человек, знакомых с~обоими, либо $17$~человек,
незнакомых с~обоими.

\item
Квадрат со~стороной~$1$ разрезали на~прямоугольники.
\emph{Шириной} прямоугольника назовем длину его меньшей стороны.
Докажите, что сумма ширин всех прямоугольников разбиения не~меньше $1$.

\item
В~волейбольном турнире сыграли $512$ команд каждая с~каждой по~одному разу.
Докажите, что можно выбрать 10 команд и~пронумеровать их числами от~$1$ до~$10$
так, что в~любой паре пронумерованных команд команда с~меньшим номеров выиграла
у~команды с~большим номером.
(В~волейболе не~бывает ничьих.)

\end{problems}

\endgroup % \def\piconst

