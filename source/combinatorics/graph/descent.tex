% $timetable:
%   g10r1:

\section*{Метод спуска в графах}

% $authors:
% - Алексей Яковлевич Канель-Белов

\claim{Лемма Холла / регулярный граф}
Пусть в~двудольном графе каждая вершина соединена ровно с~$k$ другими.
Тогда можно разбить граф на~пары вершин, соединенных ребром.

\claim{Лемма Холла}
Пусть дан двудольный граф.
Раскрасим его в~черный и~белый цвета правильно (так, чтобы вершины одинакового
цвета не~соединялись).
Пусть для любого набора черных вершин верно, что количество соединенных с~ними
белых вершин не~меньше количества черных вершин в~наборе;
то~же самое для белых вершин.
Тогда можно разбить граф на~пары вершин, соединенных ребром.

\claim{Лемма об эпидемии}
Пусть в~условиях лемма Холла (любого из~вариантов) мы выделили несколько пар
вершин, но~несколько вершин осталось.
Тогда можно добавить еще одну пару.

\claim{Теорема Менгера}
Пусть дан граф, и~в~нем отмечены две вершины $A$ и~$B$, не~соединенные ребром,
но~лежащие в~одной компоненте связности.
Пусть также при удалении любых $(k - 1)$ других вершин вершины $A$ и~$B$
остаются связными.
Тогда в~графе есть $k$ непересекающихся путей, соединяющих $A$ и~$B$.

\definition
\emph{Частично упорядоченное множество}~--- это множество, некоторые элементы
которого сравнимы, причем выполнена транзитивность
($A > B, B > C \Rightarrow A > C$).

\emph{Замечание.}
Например, целые числа~--- это частично упорядоченное множество относительно
делимости (не~все числа делятся друг на~друга, но~если $a \mid b$ и~$b \mid c$,
то~$a \mid c$).

\definition
\emph{Цепочка} в~частично упорядоченном множестве~--- это набор элементов,
среди которых любые два сравнимы.
\emph{Антицепочка}~--- это, наборот, набор элементов, среди которых никакие два
не~сравнимы.

\emph{Замечание.}
Например, в~множестве целых чисел, упорядоченном относительно делимости,
степени двойки~--- это цепочка, а~простые числа~--- антицепочка.

\emph{Замечание.}
Несложно заметить, что цепочка и~антицепочка не~могут пересекаться более чем
по~одному элементу.

\definition
\emph{Диаметр} частично упорядоченного множества~--- это максимальный размер
антицепочки в~нем.

\emph{Замечание.}
Мы не~можем разбить частично упорядоченное множество меньше, чем на~$d$
цепочек, где $d$~--- это диаметр.
Действительно, пусть это не~так.
Тогда антицепочка размера $d$ пересекается с~одной из~цепочек более, чем
по~одному элементу.

\claim{Теорема Дилворса\footnote{Robert P. Dilworth}}
Пусть диаметр частично упорядоченного множества равен $d$.
Тогда это множество можно разбить на~$d$~цепочек.

\begin{problems}

\item
В~стране несколько городов, и~между ними проложены непересекающиеся дороги.
Каждая дорога покращена в~один из~трех цветов: красный, синий, белый.
В~каждом городе сходятся ровно по~одной дороге каждого цвета.
Города делятся на~два типа: те, в~которых дороги сходятся в~порядке
<<белый--синий--красный>>, считая против часовой стрелки, и~наоборот,
<<белый--красный--синий>>.
Доказать, что разность между количеством городов двух типов делится на~$4$.

\item
Между точками $A$ и~$B$ проложена сеть трубопроводов.
Мы можем рассмотреть сечения этой сети, то~есть рассечь трубопроводы
в~нескольких местах так, чтобы вода не~текла.
Найдем сечение, через которое можно прокачать (в~сумме) минимальное количество
воды.
Докажите, что между пунктами $A$ и~$B$ можно прокачать столько~же воды, сколько
и~через минимальное сечение.

\end{problems}

