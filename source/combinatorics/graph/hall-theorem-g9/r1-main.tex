% $date: 2016-03-29
% $timetable:
%   g9r1:
%     2016-03-29:
%       1:

\section*{Лемма Холла}

% $authors:
% - Юлий Васильевич Тихонов
% на основе материалов В. З. Шарича и В. В. Трушкова.

\claim{Лемма Холла}
Есть $n$~юношей и~несколько девушек.
Известно, что каких~бы $k$~юношей ни~выбрать, число знакомых им совокупности
девушек не~меньше $k$.
Тогда все юноши могут выбрать по~невесте из~числа своих знакомых.

\begingroup
    \def\abs#1{\lvert #1 \rvert}

\begin{problems}

\item \emph{Метод чередующихся цепей.}
Докажите лемму Холла, используя следующую конструкцию:
\begin{list}{}{}\item
    Предположим, некоторые пары заключили брак, но~при этом один из~юношей
    остался неженатым.
    Этот юноша с~горя вступает в~тайное общество.
    Далее каждый молодой человек приводил в~тайное общество всех знакомых
    девушек, а~каждая молодая жена~--- своего мужа\ldots
\end{list}

\item \emph{Критические множества.}
Назовем множество из $k$~юношей \emph{критическим,} если совокупное количество
знакомых им девушек в точности ровно $k$.
\\
\subproblem
Предположим, что критическое множество юношей не содержит меньших критических
подмножеств.
Докажите, что никакая свадьба юноши из этого множества не испортит для
остальных условие леммы Холла.
\\
\subproblem
Докажите, что если удалить критическое множество юношей, вместе с их знакомыми
девушками, то для оставшихся будет выполнено условие леммы Холла.
\\
\subproblem
Докажите лемму Холла, рассмотрев критическое подмножество юношей.

\item \emph{Лемма Холла.}
Пусть имеется набор конечных множеств $S_1, S_2, \ldots, S_n$.
\emph{Системой различных представителей} называется набор
$x_1, x_2, \ldots, x_n$, $x_i \in S_i$ такой, что
\(
    i \neq j
\;\Rightarrow\;
    x_i \neq x_j
\).
Система различных представителей существует тогда и~только тогда, когда
\[
    \forall k \geq 1
\;
    \forall i_1, \ldots, i_k
\quad
    \abs{S_{i_1} \cup \ldots \cup S_{i_k}} \geq k
\, . \]

\end{problems}

\definition
\emph{Латинским прямоугольником} называется числовой прямоугольник (таблица)
$n \times m$, $n \geq m$, в~который записаны числа $1, 2, \ldots, n$, причем
в~любой строчке и~любом столбце нет равных чисел.

\begin{problems}

\item
Докажите, что латинский прямоугольник всегда можно дополнить до~латинского
квадрата.

\item
В~кубе $8 \times 8 \times 8$ несколько нижних слоев заполнены не~бьющими друг
друга ладьями.
(Заполнены~--- это значит, что в~каждом слое находится максимальное число
ладьей~--- $8$.)
Докажите, что можно заполнить оставшиеся слои с~сохранением того свойства, что
ладьи не~бьют друг друга.

\item
Труппа из~$16$~человек играет пьесу, в~которой $16$~ролей.
Один актер в~каждом спектакле играет ровно одну роль, и~роли у~каждого актера
не~повторяются в~одном сезоне.
Труппа заканчивает сезон, когда нельзя подобрать роли актерам, удовлетворяющие
этому требованию.
Может~ли труппа закончить сезон быстрее, чем за~$16$~спектаклей?
%\emph{(ФЮМ, 1995 г.)}

\item
Квадратный лист бумаги разбит на~сто многоугольников одинаковой площади с~одной
стороны и~на~сто других той~же площади с~обратной стороны.
Докажите, что этот квадрат можно проткнуть ста иголками так, что каждый
из~двухсот многоугольников проткнут по~разу.

\end{problems}

\definition
\emph{Паросочетанием} называется набор ребер~$P$, в~котором никакие два ребра
не~имеют общих вершин.
\emph{1-фактор}~--- это паросочетание, в котором участвуют все вершины графа.

\begin{problems}

\item
Докажите, что в~регулярном двудольном графе есть 1-фактор.
(У \emph{регулярного графа} степени всех вершин одинаковы.)

\item
Докажите, что регулярный двудольный граф разбивается на~1-факторы.

\item\emph{Теорема Петерсена.}
Дан регулярный граф четной степени.
\\
\subproblemx{$^\circ$}
Докажите, что в~таком графе есть цикл, проходящий по~всем ребрам ровно
по~одному разу.
\\
\subproblem
Ориентируем граф так, чтобы цикл из предыдущего пункта был ориентированным.
<<Раздвоим>> каждую вершину так, что исходящие ребра останутся у одной вершины,
а входящие~--- у другой.
Докажите, что в~получившемся двудольном графе есть 1-фактор.
\\
\subproblem
Докажите, что все \emph{вершины} исходного графа можно покрыть несколькими
непересекающимися (по вершинам) циклами.

\item
Докажите, что если любые $k$ ($1 \leq k \leq n$) юношей знакомы в~совокупности
не~менее чем с~$(k - d)$ девушками, то~$(n - d)$ юношей могут выбрать себе
невесту из~числа знакомых.

\end{problems}

\endgroup % \def\abs

