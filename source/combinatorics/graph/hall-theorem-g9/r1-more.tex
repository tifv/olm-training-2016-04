% $date: 2016-03-30
% $timetable:
%   g9r1:
%     2016-03-30:
%       3:

\section*{Лемма Холла. Добавка}

% $authors:
% - Юлий Васильевич Тихонов
% на основе материалов В. З. Шарича и В. В. Трушкова.

\setproblem{11}

\begingroup
    \def\abs#1{\lvert #1 \rvert}

\begin{problems}

\item
\emph{Лемма Холла для арабских стран.}
В~двудольном графе (доли $U$ и~$V$) каждое множество $A \subset U$ соединено
в~совокупности с~не~менее чем $r \cdot \abs{A}$ вершинами из~$V$
(фиксировано $r \in \mathbb{N}$).
Докажите, что можно выбрать $r$ паросочетаний, каждое из~которых покрывает все
вершины доли~$U$, так что каждая вершина доли~$V$ покрыта не~более чем одним
паросочетанием.

\item
Докажите, что из~$52$ натуральных чисел, не~превосходящих $100$, можно выбрать
$6$~чисел, любая пара из~которых отличается в~обоих разрядах.

\item
Есть $n$~юношей и~$n$~девушек.
Каждый юноша знает хотя~бы одну девушку.
Докажите, что тогда можно некоторых юношей поженить на~знакомых девушках так,
чтобы женатые юноши не~знали незамужних девушек.

\item
Пусть выполнено условие леммы Холла, и~каждый из~$n$~юношей знаком по~меньшей
мере с~$k$~девушками.
Докажите, что супружеские пары могут быть составлены по~крайней мере
$k!$~способами, если $k \leq n$.

\end{problems}

\endgroup % \def\abs

