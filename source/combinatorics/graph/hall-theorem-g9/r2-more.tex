% $date: 2016-03-30
% $timetable:
%   g9r2:
%     2016-03-30:
%       1:

\section*{Лемма Холла. Добавка}

% $authors:
% - Юлий Васильевич Тихонов
% на основе материалов В. З. Шарича и В. В. Трушкова.

\setproblem{11}

\begingroup
    \def\abs#1{\lvert #1 \rvert}

\begin{problems}

\item
В~некотором районе, состоящем из~нескольких деревень, число женихов равно числу
невест.
Известно, что в~каждой из~деревень общее число женихов и~невест не~превосходит
половины от~числа всех женихов и~невест всего района.
Докажите, что всех этих молодых людей можно поженить так, что в~каждой паре муж
и~жена будут из~разных деревень.

\item
Есть $n$~юношей и~$n$~девушек.
Каждый юноша знает хотя~бы одну девушку.
Докажите, что тогда можно некоторых юношей поженить на~знакомых девушках так,
чтобы женатые юноши не~знали незамужних девушек.

\item
Пусть выполнено условие леммы Холла, и~каждый из~$n$~юношей знаком по~меньшей
мере с~$k$~девушками.
Докажите, что супружеские пары могут быть составлены по~крайней мере
$k!$~способами, если $k \leq n$.

\end{problems}

\endgroup % \def\abs

