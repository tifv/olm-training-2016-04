% $date: 2016-04-02
% $timetable:
%   g10r1:
%     2016-04-02:
%       2:

\section*{Графы. Пути и раскраски}

% $authors:
% - Владимир Алексеевич Брагин
% - Иван Викторович Митрофанов

\begin{problems}

\item
\subproblem
В~футбольном турнире участвует $20$~команд.
После того, как все команды провели по~две игры, организаторы турнира решили
разбить их на~три дивизиона, но~так, чтобы в~одном дивизионе не~было команд,
уже игравших друг с~другом.
Всегда~ли они смогут это сделать?
\\
\subproblem
Докажите, что если до~этого команды дважды разбивались на~$10$ пар для того,
чтобы сыграть эти две игры, то~хватит и~двух дивизионов.

\item
За~круглым столом сидят $100$ представителей $50$~стран, по~двое от~каждой
страны.
Докажите, что их можно разбить на~две группы таким образом, что в~каждой группе
будет по~одному представителю от~каждой страны, и~каждый человек находился
в~одной группе не~более чем с~одним своим соседом.

\item
За~круглым столом сидят $100$ представителей $25$ стран, по~$4$ представителя
от~каждой.
Докажите, что их можно разбить на~$4$ группы таким образом, что в~каждой группе
будет по~одному представителю от~каждой страны, и~никакие двое из~одной группы
не~сидят за~столом рядом.

%\item
%В~школе некоторые учащиеся болеют за~какую-нибудь хоккейную команду, некоторые
%болеют за~какую-нибудь футбольную команду.
%За~каждую команду болеет не~более $100$ человек.
%Докажите, что можно раздать каждому ученику школы футболку одной из~$100$
%расцветок так, чтобы у~любых двоих, болеющих за~одну команду, были футболки
%разных цветов.

\item
Ребра графа покрашены в~два цвета так, что не~существует одноцветных нечетных
циклов.
Докажите, что вершины графа можно правильным образом покрасить в~$4$ цвета.

\item
В~стране $2016$ городов, некоторые пары городов соединены дорогами.
Известно, что через любой город проходит не~более $N$ различных
несамопересекающихся циклических маршрутов нечетной длины.
Докажите, что страну можно разделить на~$2 N + 2$ республик так, чтобы
никакие два города из~одной республики не~были соединены дорогой.

\item
Степень каждой вершины графа не~превосходит $n$.
Докажите, что можно покрасить вершины в~$n^2 + 1$ цветов так, чтобы между
одноцветными вершинами не~было пути короче $3$.
%\\
%\subproblem
%Верно~ли, что для всех $n$ можно уменьшить количество цветов?

\item
В~ориентированном графе из~каждой вершине выходит три ребра.
\\
\subproblem
В~какое наименьшее число цветов можно покрасить вершины этого графа, чтобы
ниоткуда не~выходило ребро в~вершину того~же цвета?
\\
\subproblem
Докажите, что можно покрасить вершины в~$400$ цветов так, чтобы ниоткуда
не~выходило ребро в~вершину того~же цвета, а~из~набора цветов, в~которые ведет
ребро из~вершины, однозначно~бы восстанавливался ее цвет.

\item
В~городе несколько площадей.
Некоторые пары площадей соединены улицами с~односторонним движением так, что
с~каждой площади можно выехать ровно по~двум улицам.
Докажите, что город можно разделить на~$1014$ районов так, чтобы улицами
соединялись только площади из~разных районов, и~для каждых двух районов все
соединяющие их улицы были направлены одинаково (либо все из~первого района
во~второй, либо наоборот).
\emph{Подсказка:
$1014 = 13 \cdot 78$, $13 = 2 \cdot (2 + 2 \cdot 2) + 1$, $78 = C_{13}^{2}$.}

\end{problems}

