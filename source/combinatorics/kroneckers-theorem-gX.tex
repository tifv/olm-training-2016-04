% $date: 2016-04-05
% $timetable:
%   gX:
%     2016-04-05:
%       2:

\section*{Теорема Кронекера}

% $authors:
% - Иван Викторович Митрофанов

\begin{problems}

\item
Завершите доказательство теоремы про разбиение прямоугольника на~квадраты.
(Если прямоугольник разбит на~квадраты, то отношение его сторон рационально.)

\item
Можно~ли число $0{,}1$ представить в~виде произведения ста положительных
конечных десятичных дробей, меньших единицы?

\item
На~прямой конечное число отрезков суммарной длиной $2{,}41$ покрашено черным,
в~одной из~черных точек сидит кузнечик.
Кузнечик умеет прыгать по~прямой на~$1$ влево или на~$\sqrt{2}$ вправо.
Докажите, что даже если он сам будет выбирать тип прыжка, он не~сможет всё
время оставаться на~черной части прямой.

\item
Кузнечик живет на~плоскости и~умеет прыгать на~вектора
$(-1, 0)$, $(0, -1)$ и~$(\sqrt{2}, \sqrt{3})$.
Докажите, что он не~сможет всё время оставаться внутри многоугольника площади
меньшей, чем $1 + \sqrt{2} + \sqrt{3}$ (многоугольник не~обязан быть выпуклым).

\item
На~окружности длины~$1$ несколько дуг покрасили в~черный цвет так, что нет двух
черных точек на~расстоянии $\sqrt{2} / 2$
(расстояние измеряется по~окружности).
Может~ли суммарная длина черных дуг быть больше $0{,}5$?
Больше $0{,}499$?

\item
На~окружности длины~$1$ несколько дуг покрасили в~черный цвет так, что нет двух
черных точек на~расстоянии $\sqrt{2} / 2$ и~$\sqrt{3} / 3$.
(расстояние измеряется по~окружности).
Может~ли суммарная длина черных дуг быть больше $0{,}499$?

\end{problems}

