% $date: 2016-04-02
% $timetable:
%   g10r2:
%     2016-04-02:
%       3:

\section*{Комбинаторика. По мотивам старых тем}

% $authors:
% - Владимир Алексеевич Брагин
% - Иван Викторович Митрофанов

\begin{problems}

\item
Вершины правильного $2016$-угольника покрашены в~синий и~красный так, что
вершин каждого цвета поровну.
Докажите, что можно выбрать равные
\\
\subproblem
треугольник с~вершинами в~красных точках и~треугольник с~вершинами в~синих
точках
\\
\subproblem
$500$-угольник с~вершинами в~красных точках и~$500$-угольник с~вершинами
в~синих точках.

\item
Назовем \emph{лабиринтом} шахматную доску $8 \times 8$, где между некоторыми
полями вставены перегородки.
Если ладья может обойти все поля, не~перепрыгивая через перегородки,
то~лабиринт называется \emph{хорошим,} иначе~--- \emph{плохим.}
Каких лабиринтов больше~--- хороших или плохих?

\item
Клетки таблицы $5 \times 7$ заполнены числами так, что в~каждом прямоугольнике
$2 \times 3$ (вертикальном или горизонтальном) сумма чисел равна нулю.
Заплатив $100$ рублей, можно выбрать любую клетку и~узнать, какое число в~ней
записано.
Какого наименьшего числа рублей хватит, чтобы наверняка определить сумму всех
чисел таблицы?

\item
$S$~--- $2016$-элементное множество, $n$~--- целое число,
и~$0 \leq n \leq 2^{2016}$.
Докажите, что все подмножества $S$ можно раскрасить в~черный и~белый цвета
с~соблюдением следующих условий:
\\
\textit{(1)} объединение любых двух белых подмножеств~--- белое;
\\
\textit{(2)} объединение любых двух черных подмножеств~--- черное;
\\
\textit{(3)} белых подмножеств ровно $n$.

\item
У~каждого из~жителей города~$N$ знакомые составляют не~менее $30\%$ населения
города.
Житель идет на~выборы, если баллотируется хотя~бы один из~его знакомых.
Докажите, что можно так провести выборы мэра города из~двух кандидатов, что
в~них примет участие не~менее половины жителей.

\item
В~городе установили автобусное сообщение.
Для этого выбрали место для $57$ автобусных остановок и~запустили несколько
маршрутов.
При этом мэр города потребовал, чтобы
\\
\textit{(1)}
каждый маршрут проходит хотя~бы по~трем остановкам;
\\
\textit{(2)}
для любых двух остановок есть ровно один маршрут, проходящий по~ним;
\\
\textit{(3)}
любые два маршрута имеют общую остановку.
Докажите, что маршрутов должно тоже $57$.

\end{problems}

