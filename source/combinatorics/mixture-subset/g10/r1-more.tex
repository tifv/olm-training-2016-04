% $date: 2016-03-30
% $timetable:
%   g10r1:
%     2016-03-30:
%       3:

\section*{Задачи про подмножества --- 2}

% $authors:
% - Иван Викторович Митрофанов
% - Владимир Алексеевич Брагин

\begin{problems}

\item
Пусть $S$~--- $2016$-элементное множество, $n$~--- целое число,
и~$0 \leq n \leq 2^{2016}$.
Докажите, что все подмножества $S$ можно раскрасить в~черный и~белый цвета
с~соблюдением следующих условий:
\\
\subproblem
объединение любых двух белых подмножеств~--- белое;
\\
\subproblem
объединение любых двух черных подмножеств~--- черное;
\\
\subproblem
белых подмножеств ровно $n$.

\item
Какое максимальное число подмножеств $n$-элементного множества можно взять так,
чтобы никакое из~них не~содержало другое?

\item
В~$n$-элементном множестве~$M$ выбрано $n$ различных подмножеств
$A_{1}$,~\ldots, $A_{n}$.
Докажите, что для некоторого $x \in M$ множества
$A_{1} \cup \{x\}$,~\ldots, $A_{n} \cup \{x\}$ также различны.

\item
На~концерте выступали $100$ певцов.
Они исполняли несколько песен, каждую песню сколько-то пели хором со~сцены,
а~остальные слушали из~зала.
При каком наименьшем количестве песен могло быть так, что  каждый слушал
каждого?

\end{problems}

