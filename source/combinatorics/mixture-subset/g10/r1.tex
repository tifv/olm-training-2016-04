% $date: 2016-03-29
% $timetable:
%   g10r1:
%     2016-03-29:
%       3:

\section*{Задачи про подмножества}

% $authors:
% - Владимир Алексеевич Брагин
% - Иван Викторович Митрофанов

\begin{problems}

\item
Депутаты Парламента образовали несколько комиссий не~более чем из~$10$~человек
каждая.
Известно, что любые $11$~комиссий имеют общего члена.
Докажите, что все комиссии имеют общего члена.

%\item
%В~$n$-элементном множестве~$M$ выбрано $n$ различных подмножеств
%$A_{1}$,~\ldots, $A_{n}$.
%Докажите, что для некоторого $x \in M$ множества
%$A_{1} \cup \{x\}$,~\ldots, $A_{n} \cup \{x\}$ также различны.

\item
В~классе организуется турнир по~перетягиванию каната: проводится серия
соревнований, в~каждом соревновании перетягивают канат какие-то две команды
(непустые непересекающиеся подмножества учащихся).
В~турнире ровно по~одному разу должны участвовать всевозможные команды, которые
можно составить из~учащихся этого класса (кроме команды всего класса).
Докажите, что каждая команда учащихся будет соревноваться с~командой всех
остальных учащихся класса.

\item
\jeolmlabel{combi/mixture-g10-subset/r1:problem:projective}%
В~городе Проективный собираются установить автобусное сообщение.
Для этого выбрали место для $n$ автобусных остановок и~запустили несколько
(больше одного) маршрутов.
При этом оказалось, что
\\
\textit{(i)}
каждый маршрут проходит хотя~бы по~трем остановкам;
\\
\textit{(ii)}
для любых двух остановок есть ровно один маршрут, проходящий по~ним;
\\
\textit{(iii)}
любые два маршрута имеют хотя~бы одну общую остановку.
\\
Докажите, что маршрутов тоже $n$.

\item
В~множестве, состоящем из~$n$ элементов, выбрано $2^{n-1}$ подмножеств, каждые
три из~которых имеют общий элемент.
Докажите, что все эти подмножества имеют общий элемент.

\item
Даны $2016$ множеств, каждое из~которых состоит из~$45$ элементов, причем
объединение любых двух множеств содержит ровно $89$~элементов.
Сколько элементов содержит объединение всех этих $2016$ множеств?

\item
При каком наименьшем $n$ для любого набора~$A$ из~$2016$ множеств найдется
такой набор~$B$ из~$n$ множеств, что каждое множество набора~$A$ является
пересечением двух различных множеств набора~$B$?

\item
В~Думе $1600$ депутатов, которые образовали $16000$ комитетов по~$80$ человек
в~каждом.
Докажите, что найдутся два комитета, имеющие не~менее четырех общих членов.

\item
Среди всех подмножеств некоторого конечного множества выделили несколько
$(2k)$-элементных ($k$~--- фиксированное натуральное число).
Известно, что в~каждом подмножестве, состоящем не~более чем
из~$(k + 1)^2$ элементов, либо не~содержится ни~одного выделенного
подмножества, либо все в~нем содержащиеся выделенные подмножества имеют общий
элемент.
Докажите, что все выделенные подмножества имеют общий элемент.

\item
Докажите, что при $n = 57$ можно организовать такую систему остановок
и~маршрутов, чтобы выполнялись все условия
задачи~\jeolmref{combi/mixture-g10-subset/r1:problem:projective}.

\iffalse
%\item
%Рассмотрим граф, у~которого вершины соответствуют всевозможным трехэлементным
%подмножествам множества $\{ 1, 2, 3, \ldots, 2^k \}$, а~ребра проводятся между
%вершинами, которые соответствуют подмножествам, пересекающимся ровно
%по~одному элементу.
%Найдите минимальное количество цветов, в~которые можно раскрасить вершины графа
%так, чтобы любые две вершины, соединенные ребром, были разного цвета.
\fi

\end{problems}

