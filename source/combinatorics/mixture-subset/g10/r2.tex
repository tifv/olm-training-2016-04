% $date: 2016-03-29
% $timetable:
%   g10r2:
%     2016-03-29:
%       1:

\section*{Задачи про подмножества}

% $authors:
% - Иван Викторович Митрофанов
% - Владимир Алексеевич Брагин

\begin{problems}

\item
Депутаты Парламента образовали несколько комиссий не~более чем из~$10$~человек
каждая.
Известно, что любые $11$~комиссий имеют общего члена.
Докажите, что все комиссии имеют общего члена.

\item
Комиссия из~$60$~человек провела $40$~заседаний, на~каждом заседании было ровно
$10$~человек.
Докажите, что какие-то два члена комиссии были вместе хотя~бы на~двух
заседаниях.

\item
В~классе организуется турнир по~перетягиванию каната: проводится серия
соревнований, в~каждом соревновании перетягивают канат какие-то две команды
(непустые непересекающиеся подмножества учащихся).
В~турнире ровно по~одному разу должны участвовать всевозможные команды, которые
можно составить из~учащихся этого класса (кроме команды всего класса).
Докажите, что каждая команда учащихся будет соревноваться с~командой всех
остальных учащихся класса.

\item
В~множестве, состоящем из~$n$ элементов, выбрано $2^{n-1}$ подмножеств, каждые
три из~которых имеют общий элемент.
Докажите, что все эти подмножества имеют общий элемент.

%\item
%В~$n$-элементном множестве~$M$ выбрано $n$ различных подмножеств
%$A_{1}$,~\ldots, $A_{n}$.
%Докажите, что для некоторого $x \in M$ множества
%$A_{1} \cup \{x\}$,~\ldots, $A_{n} \cup \{x\}$ также различны.

\item
Какое наибольшее количество трехэлементных подмножеств $2017$-элементного
множества можно выбрать так, чтобы никакие два из~них не~пересекались ровно
по~одному элементу?

\item
Даны $2016$ множеств, каждое из~которых состоит из~$45$ элементов, причем
объединение любых двух множеств содержит ровно $89$~элементов.
Сколько элементов содержит объединение всех этих $2016$ множеств?

\item
При каком наименьшем $n$ для любого набора~$A$ из~$2016$ множеств найдется
такой набор~$B$ из~$n$ множеств, что каждое множество набора~$A$ является
пересечением двух различных множеств набора~$B$?

%\item
%В~Думе $1600$ депутатов, которые образовали $16000$ комитетов по~$80$ человек
%в~каждом.
%Докажите, что найдутся два комитета, имеющие не~менее четырех общих членов.

%\item
%Часть подмножеств некоторого конечного множества выделена.
%Каждое выделенное подмножество состоит в~точности из~$2 k$ элементов
%($k$~--- фиксированное натуральное число).
%Известно, что в~каждом подмножестве, состоящем не~более чем
%из~$(k + 1)^2$ элементов, либо не~содержится ни~одного выделенного
%подмножества, либо все в~нем содержащиеся выделенные подмножества имеют общий
%элемент.
%Докажите, что все выделенные подмножества имеют общий элемент.

\end{problems}

