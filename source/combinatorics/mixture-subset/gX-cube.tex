% $date: 2016-03-31
% $timetable:
%   gX:
%     2016-03-31:
%       3:

\section*{Кубукция}

% $authors:
% - Иван Викторович Митрофанов

\begin{problems}

\item
Пусть $S$~--- $2016$-элементное множество, $n$~--- целое число,
и~$0 \leq n \leq 2^{2016}$.
Докажите, что все подмножества $S$ можно раскрасить в~черный и~белый цвета
с~соблюдением следующих условий:
\\
\textit{(а)} объединение любых двух белых подмножеств –- белое;
\\
\textit{(б)} объединение любых двух черных подмножеств –- черное;
\\
\textit{(в)} белых подмножеств ровно $n$.

\item
При каких $n$ все подмножества $(2^n - 2)$ множеств из~$n$ элементов, кроме
пустого и~всего, можно разбить на~пары с~симметрической разностью из~одного
элемента?

\item
Дано $n > 1$ различных натуральных чисел.
На~доску выписали все их попарные НОКи и~НОДы.
Какое могло получиться наименьшее количество чисел?

\item
В~некотором конечном множестве взяли $2^k + 1$ различных подмножеств
и~покрасили их в~два цвета, причем в~каждый цвет покрашено хотя~бы одно
множество.
Докажите, что среди симметрических разномтей разноцветных множеств есть хотя~бы
$2^k$ различных.

\item
Пусть $A$~--- такое семейство подмножеств конечного множества, что вместе
с~каждым множеством $M$ из~$A$ в~$A$ содержатся все подмножества $M$
(включая пустое множество).
Докажите, что сущетвует такое биективное отображение семейства $A$ в~себя, что
для любого $M \in A$ выполнено $M \cap f(M) = \varnothing$.

\item
Пусть $P$ и~$Q$ -- наборы слов длины~$n$ над алфавитом из~$k$ символов.
Известно, что любое слово из~$P$ и~любое слово из~$Q$ совпадают хотя~бы в~одной
позиции.
Докажите, что либо в~$P$, либо в~$Q$ не~более, чем $k^{n-1}$ разных слов.

\item
Все $512$ подмножеств $9$-ти элементного множества разбили на~пары.
Докажите, что эти $512$ подмножеств можно расставить по~кругу так, чтобы два
подмножества из~каждой пары стояли рядом и~чтобы любые два стоящие рядом
подмножества, не~принадлежащие одной паре, имели~бы симметрическую разность
мощности $1$.

\end{problems}

