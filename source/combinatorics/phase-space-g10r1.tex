% $date: 2016-04-05
% $timetable:
%   g10r1:
%     2016-04-05:
%       1:

\section*{Фазовое и конфигурационное пространство}

% $authors:
% - Владимир Алексеевич Брагин
% - Иван Викторович Митрофанов

\begin{problems}

\item
Из~Москвы в~Истру вышли $11$~путников, они шли с~постоянными, но~разными
скоростями и~стартовали (а~также финишировали) в~разное время.
Во~время их пути вдоль дороги туда-сюда летал голубь-шпион, который
в~произвольные моменты времени записывал в~свой блокнот множество путников,
которые находится ближе к~Истре, чем сам голубь.
Какое наибольшее количество различных записей могло появиться у~голубя
в~блокноте?

\item
Из~города~\text{А} в~город~\text{Б} ведут две дороги
(кривые линии, возможно, пересекающиеся, самопересекающиеся и~т.~п.).
Из~\text{А} в~\text{Б} выехали по~разным дорогам две машины, связанные веревкой
длиной $10\,\text{м}$;
оказалось, что они могут проехать из~\text{А} в~\text{Б} так, чтобы веревка
не~порвалась
(т.~е. в~любой момент расстояние между машинами было не~больше $10\,\text{м}$).
\par
Теперь возьмем два воза, представляющие собой круги диаметром $11\,\text{м}$.
Могут~ли они проехать по~этим двум дорогам, один~--- из~\text{А} в~\text{Б},
другой из~\text{Б} в~\text{А}
(каждый по~своей дороге, и~центр воза всегда находится на~дороге),
и~при этом ни~в~какой момент не~столкнуться?

\item
Дано несколько параллельных отрезков, причем для любых трех из~них найдется
прямая, их пересекающая.
Докажите, что найдется прямая, пересекающая все отрезки.

\item
Предположим, что мудрецы разбиты на~две группы, и~на~лбу каждого мудреца
написана одна из~цифр.
Каждый мудрец видит цифры мудрецов из~другой группы и~не~видит свою цифру, все
одновременно пытаются угадать свою цифру, предварительно обсудив стратегию.
Достаточно, чтобы угадал хотя бы один из них, хотя бы в одной группе.
\\
\subproblem
Докажите, что если в~одной из~групп 8 мудрецов или меньше, то~у~них это
не~получится.
\\
\subproblem
Докажите, что если в~одной из~групп 9 мудрецов, а~в~другой~--- $10^{10^9}$,
то~у~них есть совместная стратегия.

\end{problems}

