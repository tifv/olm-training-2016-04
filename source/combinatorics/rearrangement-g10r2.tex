% $date: 2016-04-05
% $timetable:
%   g10r2:
%     2016-04-05:
%       3:

\section*{Упорядочивание}

% $authors:
% - Владимир Алексеевич Брагин
% - Иван Викторович Митрофанов

\begin{problems}

\item
Из~целых чисел от~0 до~1000 выбрали 101 число.
Докажите, что среди модулей их попарных разностей есть десять различных чисел,
не~превосходящих 100.

\item
В~наборе несколько гирь, все веса которых различны.
Известно, что если положить любую пару гирь на~левую чашу, можно весы
уравновесить, положив на~правую чашу одну или несколько гирь из~остальных.
Найдите наименьшее возможное число гирь в~наборе.

\item
2000 яблок лежат в~нескольких корзинах.
Разрешается убирать корзины и~вынимать яблоки из~корзин.
Доказать, что можно добиться того, чтобы во~всех оставшихся корзинах было
поровну яблок, а~общее число яблок было не~меньше 100.


\item
На~столе лежат две кучки монет.
Известно, что суммарный вес монет из~первой кучки равен суммарному весу монет
из~второй кучки, а~для каждого натурального числа $k$, не~превосходящего числа
монет как в~первой, так и~во~второй кучке, суммарный вес $k$ самых тяжелых
монет из~первой кучки не~больше суммарного веса $k$ самых тяжелых монет
из~второй кучки.
Докажите, что если заменить каждую монету, вес которой не~меньше $x$, на~монету
веса $x$ (в~обеих кучках), то~первая кучка монет окажется не~легче второй,
каково~бы ни~было положительное число x.

\item
На~кольцевом треке $2N$ велосипедистов стартовали одновременно из~одной точки
и~поехали с~постоянными различными скоростями (в~одну сторону).
Если после старта два
велосипедиста снова оказываются одновременно в~одной точке, назовём это
встречей.
До~полудня каждые два велосипедиста встретились хотя~бы раз, при этом никакие
три или
больше не~встречались одновременно.
Докажите, что до~полудня у~каждого велосипедиста
было не~менее $N^2$ встреч.

\item
В~стране 2016 городов, любые два города соединены дорогой, длины всех дорог
попарно различны.
Докажите, что путник может выйти из~одного из~городов и~ссовершить путешествие,
пройдя по~2015 дорогам так, чтобы длина каждой следующей дороги была меньше,
чем предыдущей.

\end{problems}

