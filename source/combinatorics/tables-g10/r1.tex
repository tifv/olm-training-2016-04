% $date: 2016-04-04
% $timetable:
%   g10r1:
%     2016-04-04:
%       3:

\section*{Таблицы}

% $authors:
% - Владимир Алексеевич Брагин
% - Иван Викторович Митрофанов

\begin{problems}

\item
В~таблицу $n \times n$ записаны $n^2$~чисел, сумма которых неотрицательна.
Докажите, что можно переставить столбцы таблицы так, что сумма $n$~чисел
по~диагонали, идущей из~левого нижнего угла в~правый верхний, будет
неотрицательна.

\item
Дана таблица $n \times n$, столбцы которой пронумерованы числами от~$1$ до~$n$.
В~клетки таблицы расставляются числа $1$,~\ldots, $n$ так, что в~каждой строке
и~в~каждом столбце все числа различны.
Назовем клетку \emph{хорошей,} если число в~ней больше номера столбца,
в~котором она находится.
При каких $n$ существует расстановка, в~которой во~всех строках одинаковое
количество хороших клеток?

\item
Назовем клетчатый прямоугольник \emph{большим,} если каждая из~его сторон
больше $100 n$.
При каком наименьшем~$k$, из~любого большого прямоугольника можно вырезать
несколько полосок $1 \times 2 n$ так, чтобы осталось не~более $k$~клеток?

\item
На~бесконечной клетчатой плоскости часть клеток отметили.
Оказалось, что в~каждом прямоугольнике $m \times n$
(расположенном или так, или так) ровно одна отмеченная клетка.
Докажите, что или $m$ делится на~$n$, или $n$ делится на~$m$.

\item
В~каждой клетке квадратной таблицы записано по~числу.
В~каждой строке сумма двух наибольших чисел равна $x$, в~каждом столбце сумма
двух наибольших чисел равна $y$.
Докажите, что $x = y$.

\item
В~каждой клетке таблицы $1000 \times 1000$ стоит $0$ или $1$.
Докажите, что можно или вычеркнуть $990$ строк так, что в~каждом столбце будет
невычеркнутая единица, или вычеркнуть $990$ столбцов так, что в~каждой строке
будет невычеркнутый нуль.

\item
В~таблице $10 \times 1024$ расставлены числа $+1$ и~$-1$, при этом все
$1024$ строки различны.
Часть чисел в~клетках заменили нулями.
Докажите, что найдутся несколько (может быть, одна) строк с~нулевой суммой.

\end{problems}

