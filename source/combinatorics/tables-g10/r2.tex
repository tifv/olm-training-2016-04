% $date: 2016-04-01
% $timetable:
%   g10r2:
%     2016-04-01:
%       2:

\section*{Таблицы}

% $authors:
% - Владимир Алексеевич Брагин
% - Иван Викторович Митрофанов

\begin{problems}

\item
Доказать, что можно так расположить числа от~$1$ до~$n^2$
в~таблицу $n \times n$, чтобы суммы чисел каждого столбца были равны.

\item
В~таблицу $n \times n$ записаны $n^2$~чисел, сумма которых неотрицательна.
Докажите, что можно переставить столбцы таблицы так, что сумма $n$~чисел
по~диагонали, идущей из~левого нижнего угла в~правый верхний, будет
неотрицательна.

\item
В~квадрате $8 \times 8$ отметили несколько клеток.
Оказалось, что в~любом прямоугольнике $2 \times 4$ ровно $a$ отмеченных клеток,
а~в~любом квадрате $3 \times 3$~--- ровно $b$ отмеченных клеток.
При каких $a$ и~$b$ такое возможно?

\item
На~бесконечной клетчатой плоскости часть клеток отметили.
Оказалось, что в~каждом прямоугольнике $m \times n$
(расположенном или так, или так) ровно одна отмеченная клетка.
Докажите, что или $m$ делится на~$n$, или $n$ делится на~$m$.

\item
Дана таблица $n \times n$, столбцы которой пронумерованы числами от~$1$ до~$n$.
В~клетки таблицы расставляются числа $1$,~\ldots, $n$ так, что в~каждой строке
и~в~каждом столбце все числа различны.
Назовем клетку \emph{хорошей,} если число в~ней больше номера столбца,
в~котором она находится.
При каких $n$ существует расстановка, в~которой во~всех строках одинаковое
количество хороших клеток?

\item
Назовем клетчатый прямоугольник большим, если каждая из~его сторон больше
$100 n$.
При каком наименьшем~$k$, из~любого большого прямоугольника можно вырезать
несколько полосок $1 \times 2 n$ так, чтобы осталось не~более $k$~клеток?

\end{problems}

