% $date: 2016-04-08
% $timetable:
%   g9r2:
%     2016-04-08:
%       1:
%   g9r1:
%     2016-04-08:
%       2:

\section*{Автомедианные треугольники}

% $authors:
% - Александр Давидович Блинков

\begin{problems}

\item
Докажите, что средний по~величине угол автомедианного треугольника
не~превосходит $60^{\circ}$.

\item
Докажите, что треугольник является автомедианным тогда и~только тогда, когда:
\\
\subproblem
треугольник, составленный из~его медиан, ему подобен;
\\
\subproblem
медианы треугольника обратно пропорциональны его высотам;
\\
\subproblem
одна из~его медиан является средним квадратичным двух других;
\\
\subproblem
вершина, середины двух сторон, сходящихся в~этой вершине, и~точка пересечения
медиан лежат на~одной окружности.

\item
Пусть медиана~$BB'$ треугольника $ABC$ пересекает окружность, описанную около
этого треугольника, в~точке~$T$.
Докажите, что $ABC$~--- автомедианный тогда и~только тогда, когда:
\\
\subproblem
$M$~--- середина~$BT$;
\\
\subproblem
точки $M$ и~$T$ симметричны относительно точки~$B'$.

\item
Докажите, что треугольник является автомедианным тогда и~только тогда, когда
середина отрезка $BM$ лежит на~окружности девяти точек треугольника $ABC$.

\item
\subproblem
Докажите, что для любого треугольника выполняется равенство
\[
    OH^2 = 9 R^2 - (a^2 + b^2 + c^2)
\, , \]
где $O$~и~$H$~--- центр его описанной окружности и~ортоцентр соответственно.
\\
\subproblem
Докажите, что треугольник является автомедианным тогда и~только тогда, когда
его прямая Эйлера перпендикулярна одной из~медиан.

\item
Длина медианы, проведенной из~вершины~$B$ треугольника $ABC$ равна~$m$.
Оказалось, что основание этой медианы, середины двух других медиан и~точка
пересечения медиан лежат на~одной окружности.
Найдите $AC$.

\item
В~треугольнике $ABC$ точки $A'$ и~$C'$~--- середины сторон $BC$ и~$AB$
соответственно.
Точки $K$ и~$L$~--- основания перпендикуляров, опущенных из~$A'$
на~$AB$ и~$AC$, а~точки $M$ и~$N$~--- основания перпендикуляров, опущенных
из~$C'$ на~$BC$ и~$AC$.
Докажите, что $KL = MN$ тогда и~только тогда, когда треугольник $ABC$
равнобедренный (с~основанием~$AC$) или автомедианный (со~средней по~длине
стороной~$AC$).

\item
Две окружности пересекаются в~точках $P$ и~$Q$, $AB$~--- их общая касательная
($A$ и~$B$~--- точки касания).
Оказалось, что $Q$~--- точка пересечения медиан треугольника $ABP$.
Докажите, что отрезок~$AB$ является средним квадратичным отрезков $AP$ и~$BP$.

\end{problems}

