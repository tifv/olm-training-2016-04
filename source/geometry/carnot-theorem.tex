% $date: 2016-04-07
% $timetable:
%   g9r2:
%     2016-04-07:
%       2:
%   g9r1:
%     2016-04-07:
%       1:

\section*{Формула Карно}

% $authors:
% - Александр Давидович Блинков

Если это не~оговорено отдельно, то~треугольник, заданный в~условии,~---
остроугольный, $r$ и~$R$~--- радиусы его вписанной и~описанной окружностей
соответственно, $p$~--- полупериметр.

\begin{problems}

\item
\subproblem
Докажите, что сумма расстояний от~вершин треугольника до~ортоцентра равна сумме
диаметров его вписанной и~описанной окружностей.
\\
\subproblem
Верно~ли это утверждение, если треугольник~--- тупоугольный?

\item
Докажите, что в~треугольнике $ABC$ выполняются неравенства:
\\[0.5ex]
\subproblem $\cfrac{AH + BH + CH}{3} \leq R$;
\qquad
\subproblem $OH \geq \cfrac{R - 2 r}{3}$;
\\[0.5ex]
($H$~--- ортоцентр, $O$~--- центр описанной окружности).

\item
\subproblem
Докажите, что
\[
    m_{a} + m_{b} + m_{c}
\leq
    \frac{9}{2} \cdot R
\, , \]
где $m_{a}$, $m_{b}$ и~$m_{c}$~--- длины медиан треугольника.
\\
\subproblem
Пусть в~треугольнике $ABC$ биссектрисы углов $A$, $B$ и~$C$ пересекают
описанную окружность в~точках $W_1$, $W_2$ и~$W_3$ соответственно.
Докажите, что
\[
    A W_1 + B W_2 + C W_3
\leq
    6{,}5 \, R - r
\, . \]

\item
\subproblem
Докажите, что для углов $\alpha$, $\beta$ и~$\gamma$ треугольника выполняется
неравенство:
\[
    \frac{3 r}{R}
\leq
    \cos(\alpha) + \cos(\beta) + \cos(\gamma)
\leq
    \frac{3}{2}
\; . \]
\\
\subproblem
Пусть $A H_{A}$, $B H_{B}$ и~$C H_{C}$~--- высоты треугольника $ABC$.
% $H$~--- его ортоцентр.
Найдите сумму диаметров окружностей, описанных около треугольников
$A H_{B} H_{C}$, $B H_{A} H_{C}$, $C H_{A} H_{B}$, если даны $R$ и~$r$.

\item
В~окружность радиуса~$R$ вписан треугольник, а~в~каждый сегмент, ограниченный
стороной треугольника и~меньшей из~дуг окружности, вписана окружность
наибольшего радиуса.
Найдите сумму диаметров трех получившихся окружностей и~радиуса окружности,
вписанной в~треугольник.

\item
\subproblem
Докажите, что в~треугольнике $ABC$ выполняется равенство:
\[
    a \cdot (O M_{B} + O M_{C})
    +
    b \cdot (O M_{C} + O M_{A})
    +
    c \cdot (O M_{A} + O M_{B})
=
    2 p R
\, , \]
где $M_A$, $M_B$, $M_C$~--- середины соответствующих сторон.
\\
\subproblem \emph{(Неравенство Эрдёша)}
Пусть $h_{a}$~--- наибольшая высота треугольника $ABC$.
Докажите, что $h_{a} \geq R + r$.

\item
\subproblem
Запишите формулу Карно для случаев прямоугольного и~тупоугольного треугольников
и~обоснуйте.
\\
\subproblem
Четырехугольник $ABCD$~--- вписанный.
Пусть $r_1$ и~$r_2$~--- радиусы окружностей, вписанных в~треугольники $ABC$
и~$ADC$, а~$r_3$ и~$r_4$~--- радиусы окружностей, вписанных в~треугольники
$ABD$ и~$CBD$.
Докажите, что $r_1 + r_2 = r_3 + r_4$.

\item
Пусть $d$, $d_1$, $d_2$ и~$d_3$~--- расстояния от~центра~$O$ окружности,
описанной около треугольника, до~центров его вписанной и~вневписанных
окружностей.
Докажите, что
\[
    R^2
=
    \frac{d^2 + d_1^2 + d_2^2 + d_3^2}{12}
\; . \]

\item
\subproblem
Докажите, что если точка принадлежит отрезку, соединяющему основания двух
биссектрис треугольника, то~сумма расстояний от~этой точки до~двух сторон
треугольника равна расстоянию от~нее до~третьей стороны.
\\
\subproblem
Пусть центр~$O$ окружности, описанной около треугольника, лежит на~отрезке,
соединяющем основания двух биссектрис.
Докажите, что расстояние от~ортоцентра треугольника до~одной из~его вершин
равно $R + r$.

\end{problems}

