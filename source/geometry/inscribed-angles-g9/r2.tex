% $date: 2016-03-30
% $timetable:
%   g9r2:
%     2016-03-30:
%       3:

\section*{Считаем углы}

% $authors:
% - Фёдор Львович Бахарев

\begin{problems}

\item
Дан выпуклый четырехугольник $ABCD$ такой, что $\angle ABC = 90^{\circ}$,
$AC = CD$ и~$\angle BCA = \angle ACD$.
Точка~$F$~--- середина отрезка~$AD$.
Отрезки~$BF$ и~$AC$ пересекаются в~точке~$L$.
Докажите, что~$BC = CL$.

\item
В~треугольнике~$ABC$ выполняется равенство $BC = 2 AC$.
На~стороне~$BC$ выбрана такая точка~$D$, что $\angle CAD = \angle CBA$.
Прямая~$AD$ пересекает биссектрису внешнего угла~$C$ в~точке~$E$.
Докажите, что $AE = AB$.

\item
Точка~$D$~--- середина основания~$AC$ равнобедренного треугольника $ABC$.
Точка~$E$~--- основание перпендикуляра, опущенного из~точки~$D$
на~сторону~$BC$.
Отрезки $AE$ и~$BD$ пересекаются в~точке~$F$.
Установите, какой из~отрезков $BF$ или $BE$ длиннее.

\item
$BM$~--- медиана остроугольного треугольника $ABC$.
Касательная в~точке~$A$ к~окружности, описанной около треугольника $ABM$,
и~касательная в~точке~$C$ к~окружности, описанной около треугольника $BCM$,
пересекаются в~точке~$D$.
Докажите, что точка~$K$, симметричная точке~$D$ относительно прямой~$AC$, лежит
на~прямой~$BM$.

\item
На~сторонах~$AB$ и~$AC$ треугольника $ABC$ выбраны точки~$K$ и~$L$
соответственно так, что $KB = LC$.
Точка~$X$ симметрична $K$ относительно середины стороны~$AC$,
точка~$Y$ симметрична $L$ относительно середины стороны~$AB$.
Докажите, что прямая, содержащая биссектрису угла~$A$, делит отрезок~$XY$
пополам.

\item
В~остроугольном треугольнике $ABC$ проведены высоты $A A_1$ и~$B B_1$.
Биссектриса угла $ACB$ пересекает эти высоты в~точках $F$ и~$L$.
Докажите, что середина отрезка~$FL$ равноудалена от~точек $A_1$ и~$B_1$.

\item
Окружность, проходящая через вершины $B$ и~$C$ прямоугольного
треугольника $ABC$, пересекает гипотенузу~$AC$ в~точке~$X$.
Касательные к~этой окружности, проведенные в~точках $X$ и~$B$, пересекаются
в~точке~$Y$.
Докажите, что точка~$Y$ лежит на~средней линии треугольника $ABC$, параллельной
стороне~$BC$, или на~ее продолжении.

\end{problems}

