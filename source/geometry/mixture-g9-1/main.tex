% $date: 2016-03-29
% $timetable:
%   g9r2:
%     2016-03-29:
%       1:
%   g9r1:
%     2016-03-29:
%       3:

\section*{Вводный разнобой}

% $authors:
% - Фёдор Львович Бахарев

\begin{problems}

\item
В~треугольнике $ABC$ на~стороне~$AC$ нашлись такие точки $D$ и~$E$, что
$AB = AD$ и~$BE = EC$ ($E$~между $A$ и~$D$).
Точка~$F$~--- середина дуги~$BC$ окружности, описанной около
треугольника $ABC$.
Докажите, что точки $B$, $E$, $D$, $F$ лежат на~одной окружности.
% Зона 99.9.2, простая совсем.

\item
Точки $A'$, $B'$ и~$C'$ симметричны вершинам $A$, $B$ и~$C$ остроугольного
треугольника $ABC$ относительно центра его описанной окружности.
Докажите, что сумма площадей треугольников $A'BC$, $AB'C$ и~$ABC'$ равна
площади треугольника $ABC$.

\item
На~стороне~$AB$ треугольника $ABC$ выбрана точка~$D$.
Окружность, описанная около треугольника $BCD$, пересекает сторону~$AC$
в~точке~$M$, а~окружность, описанная около треугольника $ACD$, пересекает
сторону~$BC$ в~точке~$N$ ($M, N \neq C$).
Пусть $O$~--- центр описанной окружности треугольника $CMN$.
Докажите, что прямая~$OD$ перпендикулярна стороне~$AB$.
% Зона 00.9.7. Возня со вписанностями, очень полезно.

\item
Окружности $S_1$ и~$S_2$ с~центрами $O_1$ и~$O_2$ соответственно пересекаются
в~точках $A$ и~$B$.
Касательные к~$S_1$ и~$S_2$ в~точке~$A$ пересекают отрезки $B O_2$ и~$B O_1$
в~точках $K$ и~$L$ соответственно.
Докажите, что $KL \parallel O_1 O_2$.
% Россия 03.9.2. Упражнение на вписанность.

\item
Дан биллиард в~форме правильного $1998$-угольника
$A_{1} A_{2} \ldots A_{1998}$.
Из~середины стороны~$A_1 A_2$ выпустили шар, который, отразившись
последовательно от~сторон
$A_{2} A_{3}$, $A_{3} A_{4}$,~\ldots, $A_{1998} A_{1}$
(по~закону <<угол падения равен углу отражения>>), вернулся в~исходную точку.
Докажите, что траектория шара~--- правильный $1998$-угольник.
% Зона 98.9.7.

\item
Через центр~$I$ вписанной в~треугольник $ABC$ окружности проведена прямая,
перпендикулярная прямой~$AI$ и~пересекающая прямую~$BC$ в~точке~$M$.
Из~точки~$I$ на~прямую~$AM$ опущен перпендикуляр~$ID$.
Докажите, что точки $A$, $B$, $C$ и~$D$ лежат на~одной окружности.

\item
Окружность, вписанная в~треугольник $ABC$, касается его сторон
$AB$, $BC$ и~$CA$ в~точках $M$, $N$ и~$K$ соответственно.
Прямая, проходящая через вершину~$A$ и~параллельная $NK$, пересекает
прямую~$MN$ в~точке~$D$.
Прямая, проходящая через $A$ и~параллельная $MN$,
пересекает прямую~$NK$ в~точке~$E$.
Докажите, что прямая~$DE$ содержит среднюю линию треугольника $ABC$.

\item
Через точку~$K$, лежащую вне окружности~$\omega$, проведены касательные $KB$
и~$KD$ к~этой окружности ($B$ и~$D$~--- точки касания) и~прямая, пересекающая
окружность в~точках $A$ и~$C$.
Биссектриса угла $ABC$ пересекает отрезок~$AC$ в~точке~$E$
и~окружность~$\omega$ в~точке~$F$.
Докажите, что $\angle FDE = 90^{\circ}$.

\end{problems}

