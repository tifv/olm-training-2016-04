% $date: 2016-03-29
% $timetable:
%   g9r1:
%     2016-03-29:
%       3:

\section*{Вводный разнобой. Добавка}

% $authors:
% - Фёдор Львович Бахарев

\setproblem{8}

\begin{problems}

\item
Внутри треугольника $ABC$ дана точка~$M$.
Обозначим через $A_1$, $B_1$, $C_1$ середины сторон $BC$, $CA$, $AB$.
Докажите, что
\(
    \min(MA, MB, MC)
\leq
    2 \max(M A_1, M B_1, M C_1)
\).

\item
На сторонах $AB$ и $AC$ нашлись такие точки $P$ и $Q$ соответственно, что
$BC = BQ = CP$.
Докажите, что касательная в точке~$A$ к описанной окружности треугольника $APQ$
перпендикулярна прямой Эйлера треугольника $ABC$.

\item
Докажите, что выпуклый многоугольник может быть разрезан непересекающимися
диагоналями на остроугольные треугольники не более, чем одним способом.
% Зона 03.9.8.

\item
Вписанная окружность треугольника $ABC$ касается сторон $AB$ и $BC$ в~точках
$P$ и $Q$.
Прямая~$PQ$ пересекает описанную окружность треугольника $ABC$ в~точках
$X$ и~$Y$.
Найдите $\angle ABC$, если $\angle XBY = 135^{\circ}$.
% Туй 04.Мл7 апгрейд.

\end{problems}

