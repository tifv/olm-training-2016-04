% $date: 2016-03-31
% $timetable:
%   g9r1:
%     2016-03-31:
%       3:

\section*{Считаем отрезки}

% $authors:
% - Фёдор Львович Бахарев

\begin{problems}

\item
Дан остроугольный треугольник $ABC$.
$B_1$, $C_1$~--- основания высот из~вершин $B$, $C$ соответственно.
Точка~$D$~--- основание перпендикуляра из~$B_1$ на~$AB$, $E$~--- точка
пересечения перпендикуляра, опущенного из~точки~$D$ на~сторону~$BC$,
с~отрезком~$B B_1$.
Докажите, что прямая~$E C_1$ параллельна~$AC$.

\item
Через центр вписанной окружности четырехугольника $ABCD$ проведена
прямая.
Она пересекает сторону~$AB$ в~точке~$X$ и~сторону~$CD$ в~точке~$Y$;
углы $\angle AXY$ и~$\angle DYX$ равны.
Докажите, что $AX / BX = CY / DY$.

\item
$AE$ и~$CD$~--- высоты остроугольного треугольника $ABC$.
Биссектриса угла~$B$ пересекает отрезок~$DE$ в~точке~$F$.
На~отрезках $AE$ и~$CD$ взяли такие точки $P$ и~$Q$ соответственно, что
четырехугольники $ADFQ$ и~$CEFP$~--- вписанные.
Докажите, что $AP = CQ$.

\item
Даны непересекающиеся окружности $S_1$ и~$S_2$ и~их общие внешние
касательные $l_1$ и~$l_2$.
На~$l_1$ между точками касания отметили точку~$A$, а~на~$l_2$~---
точки $B$ и~$C$ так, что $AB$ и~$AC$~--- касательные к~$S_1$ и~$S_2$.
Пусть $O_1$ и~$O_2$~--- центры окружностей $S_1$ и~$S_2$, а~$K$~--- точка
касания вневписаной окружности треугольника $ABC$ со~стороной~$BC$.
Докажите, что середина отрезка~$O_1 O_2$ равноудалена от~точек $A$ и~$K$.

\item
Пусть $O$~--- центр описанной окружности остроугольного неравнобедренного
треугольника~$ABC$, точка~$C_1$ симметрична $C$ относительно $O$,
$D$~--- середина стороны~$AB$, $K$~--- центр описанной окружности
треугольника $O D C_1$.
Докажите, что точка~$O$ делит пополам отрезок прямой~$OK$, лежащий внутри
угла $ACB$.

\item
В~остроугольном треугольнике $ABC$ с~наименьшей стороной~$AC$ отметили
точки $P$ и~$Q$ на~сторонах $AB$ и~$BC$ соответственно так, что $AP = CQ = AC$.
Докажите, что прямая~$PQ$ перпендикулярна прямой~$OI$, где $O$ и~$I$~---
центры описанной и~вписанной окружности треугольника.

\item
Чевианы $AM$, $BN$ и~$CK$ треугольника $ABC$ пересекаются в~точке~$O$.
Оказалось, что четырехугольники $ANOK$ и~$BKOM$ являются описанными.
Докажите, что четырехугольник $CMON$ также является описанным.

\item
Медиана~$AM$ треугольника $ABC$ пересекает вписанную в~него окружность
в~точках $X$ и~$Y$.
Известно, что $AB = AC + AM$.
Найдите $\angle XIY$, где $I$~--- центр вписанной окружности.

\item
Вписанная в~треугольник $ABC$ окружность касается сторон $BC$, $CA$ и~$AB$
в~точках $A_1$, $B_1$ и~$C_1$ соответственно.
Через точку~$A_1$ проведена прямая~$\ell$, перпендикулярная отрезку~$A A_1$.
Она пересекается с~прямой~$B_1 C_1$ в~точке~$X$.
Докажите, что прямая~$BC$ делит отрезок~$AX$ пополам.

\item
На~сторонах $BC$ и~$AB$ треугольника $ABC$ стоят точки $X$ и~$Y$ так, что
$\angle BAX = \angle BCY = \alpha$.
Из~вершины~$B$ опущены перпендикуляры $BK$ и~$BL$ на~отрезки $AX$ и~$BY$
соответственно.
Найдите углы треугольника $KLM$, где $M$~--- середина стороны~$AC$.

\end{problems}

