% $date: 2016-04-04
% $timetable:
%   g9r1:
%     2016-04-04: {}

\section*{Разнобой}

% $authors:
% - Фёдор Львович Бахарев

\begin{problems}

\item
Стороны четырехугольников $ABCD$ и~$A_1 B_1 C_1 D_1$ соответственно равны,
причем $AB \parallel CD$, $B_1 C_1 \parallel A_1 D_1$.
Докажите, что оба четырехугольника~--- параллелограммы.

\item
Отрезки, соединяющие точку~$K$ с~вершинами $A$ и~$D$ прямоугольника $ABCD$
пересекают сторону~$BC$.
Перпендикуляры, опущенные из~$B$ и~$C$ на~прямые~$DK$ и~$AK$ соответственно
пересекаются в~точке~$M$.
Докажите, что если $M$ не~совпадпет с~$K$, то~$MK$ перпендикулярно $AD$.

\item
Диагонали вписанного четырехугольника $ABCD$ пересекаются в~точке~$M$,
$\angle AMB = 60^{\circ}$.
На~сторонах $AD$ и~$BC$ во~внешнюю сторону построены равносторонние
треугольники $ADK$ и~$BCL$.
Прямая~$KL$ пересекает описанную около $ABCD$ окружность в~точках $P$ и~$Q$.
Докажите, что $PK = LQ$.

\item
Дана окружность~$\omega$ и~точка~$P$ вне нее.
Проходящая через $P$ прямая~$\ell$ пересекает $\omega$ в~точках $A$ и~$B$.
На~отрезке~$AB$ отмечена точка~$C$ такая, что $PA \cdot PB = PC^2$.
Точки $M$ и~$N$~--- середины двух дуг, на~которые хорда~$AB$ разбивает
окружность~$\omega$.
Докажите, что величина $\angle MCN$ не~зависит от~выбора прямой~$\ell$.

\item
В~треугольнике $ABC$ проведена биссектриса~$B B_1$.
Перпендикуляр из~$B_1$ на~$BC$ пересекает дугу~$BC$ описанной окружности
треугольника $ABC$ в~точке~$K$.
Перпендикуляр из~$B$ на~$AK$ пересекает $AC$ в~точке~$L$.
Докажите что точки $K$, $L$  и~середина дуги~$AC$ (не~содержащей точку~$B$)
лежат на~одной прямой.

\item
Стороны треугольника $ABC$ видны из~точки~$T$ под углами $120^\circ$.
Докажите, что прямые, симметричные прямым $AT$, $BT$ и~$CT$ относительно прямых
$BC$, $AC$ и~$AB$ соответственно, пересекаются в~одной точке.

\item
В~ромб $ABCD$ вписана окружность~$\omega$.
Прямая~$\ell$ касается $\omega$ и~пересекает стороны $AB$ и~$AD$ в~точках $K$
и~$L$, а~продолжения сторон $BC$ и~$CD$ в~точках $M$ и~$N$ соответственно.
Докажите, что произведение площадей треугольников $AKL$ и~$CMN$ не~зависит
от~положения прямой~$\ell$.

\item
На~дуге $AC$ описанной окружности треугольника $ABC$ взята произвольная
точка~$P$.
Пусть $I_1$ и~$I_2$~--- центры вписанных окружностей треугольников
$ABP$ и~$CBP$.
Докажите, что описанная окружность треугольника $I_1 I_2 P$ проходит через
некоторую фиксированную точку, не~зависящую от~выбора $P$.

\end{problems}

