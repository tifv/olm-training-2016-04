% $date: 2016-04-04
% $timetable:
%   g9r2:
%     2016-04-04: {}

\section*{Разнобой}

% $authors:
% - Фёдор Львович Бахарев

\begin{problems}

\item
На~сторонах $AB$ и~$BC$ остроугольного треугольника $ABC$ построены как
на~основаниях равнобедренные треугольники $AFB$ и~$BLC$, причем один из~них
лежит внутри треугольника $ABC$, а~другой построен во~внешнюю сторону.
При этом $\angle AFB = \angle BLC$ и~$\angle CAF = \angle ACL$.
Докажите, что прямая~$FL$ отсекает от~угла
$ABC$ равнобедренный треугольник.

\item
На~стороне~$AC$ треугольника $ABC$ отмечена точка~$D$.
Произвольный луч~$\ell$, выходящий из~вершины~$B$, пересекает отрезок~$AC$
в~точке~$K$, а~описанную окружность треугольника $ABC$~--- в~точке~$L$.
Докажите, что описанная окружность треугольника $DKL$ проходит через
фиксированную точку, отличную от~$D$ и~не~зависящую от~выбора луча~$\ell$.

\item
Через вершину~$A$ остроугольного треугольника $ABC$ проведены касательная~$AK$
к~его описанной окружности, а~также биссектрисы $AN$ и~$AM$ внутреннего
и~внешнего углов при вершине~$A$ (точки $M$, $K$ и~$N$ лежат на~прямой~$BC$).
Докажите, что $MK = KN$.

\item
В~четырехугольнике $ABCD$ углы $A$ и~$C$~--- прямые.
На~сторонах $AB$ и~$CD$ как на~диаметрах построены окружности, пересекающиеся
в~точках $X$ и~$Y$.
Докажите, что прямая~$XY$ проходит через середину диагонали~$AC$.

\item
Точки $A_2$, $B_2$ и~$C_2$ --– середины высот $A A_1$, $B B_1$ и~$C C_1$
остроугольного треугольника $ABC$.
Найдите сумму углов
$\angle B_2 A_1 C_2$, $\angle C_2 B_1 A_2$ и~$\angle A_2 C_1 B_2$.

\item
Дана окружность~$\omega$ и~точка~$P$ вне нее.
Проходящая через $P$ прямая~$\ell$ пересекает $\omega$ в~точках $A$ и~$B$.
На~отрезке~$AB$ отмечена точка~$C$ такая, что $PA \cdot PB = PC^2$.
Точки $M$ и~$N$~--- середины двух дуг, на~которые хорда~$AB$ разбивает
окружность~$\omega$.
Докажите, что величина $\angle MCN$ не~зависит от~выбора прямой~$\ell$.

\end{problems}

