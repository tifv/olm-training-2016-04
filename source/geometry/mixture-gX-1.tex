% $date: 2016-03-29
% $timetable:
%   gX:
%     2016-03-29:
%       2:

\section*{Геометрический блиц}

% $authors:
% - Фёдор Львович Бахарев

\begin{problems}

\item
В~четырехугольнике $ABCD$ выполнены равенства $AB = BC$, $CD = DA$.
Отрезки~$DK$ и~$BL$~--- высоты треугольников $ABD$ и~$BCD$, соответственно.
Докажите, что прямая~$KL$ проходит через точку пересечения диагоналей
четырехугольника $ABCD$.

\item
Высоты $A A_1$ и~$C C_1$ остроугольного треугольника $ABC$ пересекаются
в~точке~$H$.
Серединные перпендикуляры к~отрезкам $AH$ и~$CH$ пересекают стороны
$AB$ и~$BC$ в~точках $X$ и~$Y$ соответственно.
Оказалось, что точка~$H$ лежит на~прямой~$XY$.
Докажите, что на~этой прямой лежит и~центр описанной окружности $ABC$.

\item
В~остроугольном треугольнике $ABC$ проведена высота~$AH$.
На~сторонах $AB$ и~$AC$ выбраны точки $P$ и~$Q$ таким образом, что $AH$~---
биссектриса угла $PHQ$, а~$\angle CQH = \angle C - \angle B$.
Докажите, что $AP + QH = PH$.

\item
Внутри треугольника $ABC$ дана точка~$M$.
Обозначим через $A_1$, $B_1$, $C_1$ середины сторон $BC$, $CA$, $AB$.
Докажите, что
\(
    \min(MA, MB, MC)
\leq
    2 \max(M A_1, M B_1, M C_1)
\).

\item
На~сторонах $AB$ и~$AC$ нашлись такие точки $P$ и~$Q$ соответственно, что
$BC = BQ = CP$.
Докажите, что касательная в~точке~$A$ к~описанной окружности треугольника $APQ$
перпендикулярна прямой Эйлера треугольника $ABC$.

\item
Стороны четырехугольников $ABCD$ и~$A_1 B_1 C_1 D_1$ соответственно равны,
причем $AB \parallel CD$, $B_1 C_1 \parallel A_1 D_1$.
Докажите, что оба четырехугольника~--- параллелограммы.

\item
Стороны треугольника $ABC$ видны из~точки~$T$ под углами $120^{\circ}$.
Докажите, что прямые, симметричные прямым $AT$, $BT$ и~$CT$ относительно прямых
$BC$, $AC$ и~$AB$ соответственно, пересекаются в~одной точке.

\item
Диагонали вписанного четырёхугольника $ABCD$ пересекаются в~точке~$M$,
$\angle AMB = 60^{\circ}$.
На~сторонах $AD$ и~$BC$ во~внешнюю сторону построены равносторонние
треугольники $ADK$ и~$BCL$.
Прямая~$KL$ пересекает описанную около $ABCD$ окружность в~точках $P$ и~$Q$.
Докажите, что $PK = LQ$.
% Устная олимпиада по~геометрии, 2004

\item
Докажите, что выпуклый многоугольник может быть разрезан непересекающимися
диагоналями на~остроугольные треугольники не~более, чем одним способом.
% Зона 03.9.8.

\item
В~треугольнике $ABC$ проведена биссектриса~$B B_1$.
Перпендикуляр из~$B_1$ на~$BC$ пересекает дугу~$BC$ описанной окружности
треугольника $ABC$ в~точке~$K$.
Перпендикуляр из~$B$ на~$AK$ пересекает $AC$ в~точке~$L$.
Докажите что точки $K$, $L$  и~середина дуги~$AC$ (не~содержащей точку~$B$)
лежат на~одной прямой.

\item
Точка~$M$~--- середина дуги $BAC$ описанной окружности треугольника $ABC$,
$I$~--- центр его вписанной окружности, $L$~--- основание биссектрисы~$AL$.
Прямая~$MI$ пересекает описанную окружность в~точке~$K$.
Описанная окружность треугольника $AKL$ пересекает прямую~$BC$ вторично
в~точке~$P$.
Докажите, что $\angle AIP = 90^{\circ}$.

\item
Вписанная окружность треугольника $ABC$ касается сторон $AB$ и~$BC$ в~точках
$P$ и~$Q$.
Прямая~$PQ$ пересекает описанную окружность треугольника $ABC$ в~точках
$X$ и~$Y$.
Найдите $\angle ABC$, если $\angle XBY = 135^{\circ}$.
% Туй 04.Мл7 апгрейд.

\end{problems}

