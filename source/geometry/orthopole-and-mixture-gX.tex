% $date:
% $timetable:
%   gX:
%     2016-03-30:
%       1:

\section*{Разнобой посложнее и ортополюсы}

% $authors:
% - Фёдор Львович Бахарев

\begin{problems}

\item
Вписаная окружность треугольника $ABC$ касается стороны~$BC$ в~точке~$A_1$.
Прямая~$A A_1$ вторично пересекает окружность в~точке~$P$.
Пусть прямые $CP$ и~$BP$ пересекают окружность повторно в~точках $N$ и~$M$
соответственно.
Докажите, что $A A_1$, $BN$ и~$CM$ пересекаются в~одной точке.
% https://www.artofproblemsolving.com/community/c6h2619p7954

\item
На~плоскости даны $n$ различных точек и~круг радиуса~$r$ с~центром в~точке~$O$.
Хотя~бы одна точка лежит внутри круга.
На~каждом шаге мы передвигаем $O$ в~центр масс точек, попавших внутрь круга.
Докажите, что процесс стабилизируется.
% https://www.artofproblemsolving.com/community/c6h34208p212361

\item
Диагонали вписанного четырехугольника $ABCD$ пересекаются в~точке~$P$.
Окружность~$\Gamma$ касается продолжений $AB$, $BC$, $AD$, $DC$ в~точках
$X$, $Y$, $Z$, $T$ соответственно.
Окружность~$\Omega$ проходит через $A$ и~$B$, и~касается внешне
окружности~$\Gamma$ в~точке~$S$.
Докажите, что $SP \perp ST$.
% http://www.artofproblemsolving.com/community/c6h1215108p6043311

\item
Пусть $H$~--- ортоцентр треугольника $ABC$, $P$~--- произвольная точка
на~окружности описанной около треугольника $ABC$, симедиана из~вершины~$A$
треугольника $APH$ пересекает $BC$ в~точке~$X$.
Симедиана из~вершины~$B$ треугольника $BPH$ пересекает $CA$ в~точке~$Y$,
симедиана из~вершины~$C$ треугольника $CPH$ пересекает $AB$ в~точке~$Z$.
Докажите, что точки $X$, $Y$ и~$Z$ коллинеарны.
% https://www.artofproblemsolving.com/community/c6t316f6h1205751

\item
Из~вершин треугольника $ABC$ опустим перпендикуляры $AA'$, $BB'$ и~$CC'$
на~прямую~$\ell$.
Докажите, что перпендикуляры из~$A'$ на~$BC$, из~$B'$ на~$CA$ и~из~$C'$ на~$AB$
пересекаются в~одной точке.
Эта точка называется \emph{ортополюсом} прямой~$\ell$ относительно
треугольника $ABC$.
% http://www.artofproblemsolving.com/community/c6h612457
% http://mathworld.wolfram.com/Orthopole.html

\item
Докажите, что ортополюс прямой~$\ell$ относительно треугольника $ABC$ лежит
на~некоторой прямой Симсона, перпендикулярной $\ell$.
Что происходит с~ортополюсом, если прямая~$\ell$ смещается параллельно самой
себе?

\item
Докажите, что если прямая~$\ell$ пересекает описанную окружность
треугольника $ABC$ в~точках $P_1$ и~$P_2$, то~прямые Симсона точек $P_1$
и~$P_2$ пересекаются в~ортополюсе прямой~$\ell$.

\item
Докажите, что если $\ell$ проходит через центр описанной окружности
треугольника $ABC$, то~ее ортополюс лежит на~окружности девяти точек
треугольника.

\item
Четырехугольник $ABCD$ вписан.
Докажите, что ортополюсы прямой~$\ell$ относительно треугольников
$ABC$, $BCD$, $CDA$ и~$DAB$ лежат на~одной прямой.

\item
Пусть $O_{a}$, $O_{b}$, $O_{c}$, $O_{d}$~--- центры описанных окружностей
треугольников $BCD$, $CDA$, $DAB$, $ABC$ соответственно.
Пусть $T_{a}$, $T_{b}$, $T_{c}$, $T_{d}$ ортополюсы прямых
$O_{a} P$, $O_{b} P$, $O_{c} P$ и~$O_{d} P$ относительно треугольников
$BCD$, $CDA$, $DAB$, $ABC$ соответственно.
Тогда точки $T_{a}$, $T_{b}$, $T_{c}$ и~$T_{d}$ лежат на~одной окружности.

\end{problems}

