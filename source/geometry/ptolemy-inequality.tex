% $date: 2016-04-06
% $timetable:
%   g9r2:
%     2016-04-06:
%       1:
%   g9r1:
%     2016-04-06:
%       2:

\section*{Теорема и неравенство Птолемея}

% $authors:
% - Александр Давидович Блинков

\begingroup
    \def\abs#1{\lvert #1 \rvert}
    \ifdefined\mathup
        \def\piconst{\mathrm{\pi}}%
    \else
        \def\piconst{\uppi}%
    \fi

\begin{problems}

\item\emph{Задача Птолемея.}
В~треугольнике $ABC$: $\abs{BC} = a$, $\abs{AC} = b$.
Найдите $\abs{AB}$, если радиус окружности, описанной около $ABC$, равен $R$.

\item
Биссектриса угла~$A$ треугольника $ABC$ пересекает описанную вокруг него
окружность в~точке~$W$.
\\
\subproblem
Выразите отношение $\abs{AW} : \abs{IW}$ через длины сторон
треугольника ($I$~--- центр окружности, вписанной в~треугольник $ABC$).
\\[0.5ex]
\subproblem
Докажите, что
\(
    \abs{AW}
>
    \cfrac{\abs{AB} + \abs{AC}}{2}
\).

% spell string Теорема Помпе\'{ю}

\item\emph{(Теорема Помпе\'{ю})}
Точка~$M$ лежит на~окружности, описанной около равностороннего
треугольника $ABC$.
\\
\subproblem
Докажите, что сумма расстояний от~$M$ до~двух вершин треугольника равна
расстоянию от~$M$ до~третьей вершины.
\\
\subproblem
Укажите все такие точки~$X$ плоскости, что из~отрезков $XA$, $XB$ и~$XC$ можно
составить треугольник.

\item
Сумма расстояний от~точки~$X$, выбранной вне квадрата, до~двух его ближайших
соседних вершин равна~$m$.
Найдите наибольшее значение суммы расстояний от~$X$ до~двух других вершин
квадрата.

\item
\subproblem
Точки $A$, $B$, $C$ и~$D$~--- четыре последовательные вершины правильного
семиугольника.
Докажите, что
\[
    \frac{1}{\abs{AB}}
=
    \frac{1}{\abs{AC}} + \frac{1}{\abs{AD}}
\; . \]
\\
\subproblem
Докажите, что
\[
    \frac{1}{\sin (\piconst / 7)}
=
    \frac{1}{\sin (2 \piconst / 7)}
    +
    \frac{1}{\sin (3 \piconst / 7)}
\; . \]

\item
В~выпуклом шестиугольнике $ABCDEF$:
$\abs{AB} = \abs{BC} = a$, $\abs{CD} = \abs{DE} = b$,
$\abs{EF} = \abs{FA} = c$.
Докажите, что
\[
    \frac{a}{\abs{BE}} + \frac{b}{\abs{AD}} + \frac{c}{\abs{CF}}
\geq
    \frac{3}{2}
\; . \]

\item
Четырехугольник $ABCD$ вписан в~окружность, $I$~--- центр окружности, вписанной
в~треугольник $ABD$.
Найдите наименьшее значение $BD$, если $AI = BC = CD = 2$.

\item
Объясните, как построить четырехугольник $ABCD$, если даны его стороны
и~известно, что он вписанный.

\item
В~остроугольном треугольнике $ABC$ обозначим $d_1$, $d_2$ и~$d_3$~---
расстояния от~центра~$O$ описанной окружности до~сторон.
Докажите, что $d_1 + d_2 + d_3 = R + r$, где $R$ и~$r$~--- радиусы описанной
и~вписанной окружностей данного треугольника.

\end{problems}

\endgroup % \def\abs \def\piconst

