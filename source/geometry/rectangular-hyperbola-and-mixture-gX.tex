% $date: 2016-04-04
% $timetable:
%   gX:
%     2016-04-04:
%       1:

\section*{Равнобокая гипербола и разнобой}

% $authors:
% - Фёдор Львович Бахарев

\begin{problems}

\item
Равнобокой гиперболой называется гипербола с~перпендикулярными асимптотами.
Докажите, что если три вершины треугольника лежат на~равнобокой гиперболе,
то~его ортоцентр тоже лежит на~этой гиперболе.

\item
Через вершины треугольника $ABC$ проведена равнобокая гипербола, пересекающая
описанную окружность в~точке~$P$, отличной от~$A$, $B$ и~$C$.
Докажите, что $P$ и~ортоцентр~$H$ симметричны относительно центра симметрии
гиперболы.

\item
Докажите, что центр симметрии равнобокой гиперболы, проходящей через вершины
треугольника, лежит на~окружности девяти точек.

\item
Пусть точка~$P$ лежит на~описанной окружности треугольника $ABC$
с~ортоцентром~$H$.
Равнобокая гипербола проходит через точку~$P$ и~вершины треугольника.
Докажите, что симедиана из~вершины~$A$ в~треугольнике $APH$ касается гиперболы.

\item
Дана точка $X$ такая, что $A$ лежит внутри треугольника $BXC$.
При этом $2 \angle XBA = \angle ACB$, $2 \angle XCA = \angle ABC$.
Докажите, что центры описанной и~вневписанной со~стороны~$BC$ окружностей
треугольника $ABC$ и~точка~$X$ лежат на~одной прямой.

\item
Четырехугольник $ABCD$ вписан в~окружность с~центром~$O$ и~диаметром~$AC$.
Касательная в~точке~$C$ к~окружности пересекает прямую~$BD$ в~точке~$P$.
Луч~$AB$ пересекает отрезок~$PO$ в~точке~$E$.
Докажите, что $\angle DCE = 90^\circ$.
% http://www.artofproblemsolving.com/community/c6h224950p1249244

\item
Дан остроугольный треугольник $ABC$.
Точка~$X$ лежит на~высоте треугольника из~вершины~$A$ (внутри треугольника),
а~точки $Y$ и~$Z$ лежат на~продолжениях высот из~вершин $B$ и~$C$
(вне треугольника).
При этом $\angle AZB = \angle BXC = \angle CYA = 90^\circ$.
Докажите, что точки $X$, $Y$ и~$Z$ коллинеарны тогда и~только тогда, когда
длина касательной из~точки~$A$ окружности девяти точек равна сумме длин
касательных из~$B$ и~$C$.
% http://www.artofproblemsolving.com/community/c6h1097301p4926185

\item
В~треугольнике $ABC$ даны центр вписанной окружности~$I$ и~центр описанной
окружности $O$.
Прямая~$\ell$ параллельна $BC$ и~касается вписаной окружности и~пересекает
прямую~$IO$ в~точке~$X$.
Точка~$Y$ на~$\ell$ такова, что $\angle YIO = 90^\circ$.
Докажите, что точки $A$, $X$, $O$, $Y$ лежат на~одной окружности.
% http://www.artofproblemsolving.com/community/q2h598536p3551853

\item
Пусть $AH$~--- высота треугольника $ABC$ и~точки~$H'$ симметрична $H$
относительно середины стороны~$BC$.
Касательные в~точках $B$ и~$C$ к~окружности, описанной около треугольника,
пересекаются в~точке~$X$.
Перпендикуляр к~прямой~$XH'$ в~точке~$H'$
пересекает прямые $AB$ и~$AC$ в~точках $Y$ и~$Z$ соответственно.
Докажите, что $\angle ZXC = \angle YXB$.
% http://www.artofproblemsolving.com/community/c6h1095220p4902494

\item
Вписаная окружность остроугольного треугольника $ABC$ имеет центр~$I$
и~касается сторон $CA$ и~$AB$ в~точках $E$ и~$F$.
Касательные к~описанной окружности треугольника $AEF$ в~точках $E$ и~$F$
пересекаются в~точке $S$.
Прямые $EF$ и~$BC$ пересекаются в~точке~$T$.
Докажите, что окружность с~диаметром~$ST$ ортогональна окружности девяти точек
треугольника $BIC$.
% http://www.artofproblemsolving.com/community/c6h1113644p5087419

%\item
%Consider a circle $(O)$ and two fixed points $B,C$ on $(O)$ such that $BC$
%is not the diameter of $(O)$.
%$A$ is an arbitrary point on $(O)$, distinct from $B,C$.
%Let $D,J,K$ be the midpoints of $BC,CA,AB$, respectively, $E,M,N$ be the feet
%of perpendiculars from $A$ to $BC$, $B$ to $DJ$, $C$ to $DK$, respectively.
%The two tangents at $M,N$ to the circumcircle of triangle $EMN$ meet at $T$.
%Prove that $T$ is a fixed point (as $A$ moves on $(O)$).
%% http://www.artofproblemsolving.com/community/c6h475439p2662677

%\item
%Let $ABC$ be a scalene triangle with circumcircle $\Omega$, and suppose
%the incircle of $ABC$ touches $BC$ at $D$.
%The angle bisector of $\angle A$ meets $BC$ and $\Omega$ at $E$ and $F$.
%The circumcircle of $\triangle DEF$ intersects the $A$-excircle at
%$S_1$, $S_2$, and $\Omega$ at $T \neq F$.
%Prove that line $AT$ passes through either $S_1$ or $S_2$.
%% http://www.artofproblemsolving.com/community/c6h1176478p5679361

\end{problems}

