% $date: 2016-03-31
% $timetable:
%   g10r2:
%     2016-03-31:
%       1:
%       2:

\section*{Тренировочная олимпиада}

% $authors:
% - Владимир Алексеевич Брагин
% - Иван Викторович Митрофанов

\begin{problems}

\item
Углы треугольника $ABC$ равны $\alpha, \beta, \gamma$.
Докажите, что
\[
    \cos\left( \frac{\alpha}{2} \right) +
    \cos\left( \frac{\beta }{2} \right) +
    \cos\left( \frac{\gamma}{2} \right)
>
    \sin\left( \frac{\alpha}{2} \right) +
    \sin\left( \frac{\beta }{2} \right) +
    \sin\left( \frac{\gamma}{2} \right)
\; . \]

\item
Назовем число \emph{прекрасным,} если в~его десятичной записи присутствуют
только цифры $0$, $3$ и~$9$.
Сколько чисел, не~превосходящих $10^{100}$, нельзя представить в~виде суммы
четырех прекрасных чисел?

\item
Окружности $\omega_1$ и~$\omega_2$ пересекаются в~точках $D$ и~$P$.
Точки $A$ и~$B$ лежат на~окружностях $\omega_1$ и~$\omega_2$ соответственно,
причем $AB$~— общая касательная к~этим окружностям, а~$D$ лежит внутри $ABP$.
$AD$ вторично пересекает окружность~$\omega_2$ в~точке $C$,
$M$~--- середина $BC$.
Докажите, что $\angle DPM = \angle BDC$.

\item
Учащиеся писали тест.
Тест состоял из~$100$ вопросов, на~каждый было два варианта ответа.
У~любых двух учащихся совпали ответы не~более, чем на~$49$~вопрос.
Докажите, что тест писало не~более $50$~человек.
%В~ботаническом справочнике каждое растение характеризуется $100$ признаками
%(каждый признак либо присутствует, либо отсутствует).
%Растения считаются непохожими, если они различаются не~менее, чем
%по~$51$~признаку.
%Покажите, что в~справочнике не~может находиться больше $50$ попарно непохожих
%растений.

\end{problems}

